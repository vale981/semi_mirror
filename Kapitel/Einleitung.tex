\section{Einleitung}
\" Aber wenn wir einen Reaktor hätten, einen Versuchsreaktor, der brennt , .... wir hätten eine Tür aufgemacht in ein völlig neues Zeitalter!" \footcite{LeschH} schrieb der Astrophysiker Harald Lesch in einem seiner Werke, um seine Meinung zur Zukunft und Bedeutung der Energiegewinnung zu verdeutlichen. Dieses Zitat verdeutlicht unsere Zielsetzung für diese Arbeit: die Aufarbeitung des Themas Kernfusion als zukünftige Form der Energiegewinnung.\\
Mit unserer Arbeit möchten wir Sie über die Kernfusion als Alternative zu den jetzigen Methoden der Energieerzeugung, insbesondere zur Kernspaltung aufmerksam machen. Besonders aufgrund der Aktualität ist dieses Thema von grundlegender Bedeutung für unsere Gesellschaft. Durch Ressourcenknappheit und Klimawandel ist eine neu Form der Energiegewinnung essenziell. Kernfusion stellt hier eine in vielen Punkten eine überlegende Methode dar, da sie sowohl eine preisgünstige als auch eine sichere und umweltverträgliche ist. \\
Jedoch findet dieses Problem in der Bevölkerung bisher leider kaum Resonanz, da Interesse der allgemeinen Bevölkerung und Publikation in Medien fehlt. Diese Aspekte motivierten uns maßgeblich zur Wahl dieses Themas. Hinzuzufügen ist unser Interesse an der Zukunft der Physik und Gesellschaft sowie unsere Neugier über die Auswirkung eines revolutionären Reaktors.\\ 
Der gesellschaftliche Teil der Arbeit beinhaltet das Abbild der Aufklärung und Meinung verschiedener Bevölkerungsschichten. Diese Arbeit basiert auf der Analyse und Auswertung einer Umfrage . Der wissenschaftliche Teil der Arbeit ist gestützt durch Literatur- und Internetquellen sowie Interviews von fachkundigen Personen im Gebiet der Physik.\\
Im nachfolgenden Teil dieser Arbeit werden wir uns mit der Geschichte der Energiegewinnung beschäftigen. Hierbei gehen wir auf Verschiedene Vergleichspunkte ein, um eine aussagekräftige Übersicht zu ermöglichen. Entscheidende Fortschritte werden dadurch deutlich. Gefolgt vom größten Teil der Arbeit die Beschreibung und Analyse der Kernfusion in wissenschaftlicher Hinsicht.\\
Anschließend wird die grundlegende Funktionsweise und der Methoden der Nutzbarmachung der Fusionsreaktion erläutert.\\
Der letzte Teil dieser Arbeit beschäftigt sich mit der aktuellen Energiesituation im Allgemeinen und den Vorteilen der Kernfusion gegenüber anderen Methoden der Energiegewinnung im Genauen.\\
