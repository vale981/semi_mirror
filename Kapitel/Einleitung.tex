\section{Einleitung}
\" Aber wenn wir einen Reaktor hätten, einen Versuchsreaktor, der brennt , .... wir hätten eine Tür aufgemacht in ein völlig neues Zeitalter!" \footcite{LeschH} schrieb der Astrophysiker Harald Lesch in einem seiner Werke, um seine Meinung zur Zukunft und Bedeutung der Energiegewinnung zu verdeutlichen. Dieses Zitat verdeutlicht unsere Zielsetzung für diese Arbeit: die Aufarbeitung des Themas Kernfusion als zukünftige Form der Energiegewinnung.\\
Mit unserer Arbeit möchten wir Sie über die Kernfusion als Alternative zu den jetzigen Methoden der Energieerzeugung, insbesondere zur Kernspaltung aufmerksam machen. Besonders aufgrund der Aktualität ist dieses Thema von grundlegender Bedeutung für unsere Gesellschaft. Durch Ressourcenknappheit und Klimawandel ist eine neue Form der Energiegewinnung essenziell. Kernfusion stellt hier eine in vielen Punkten eine überlegende Methode dar, da sie sowohl eine preisgünstige als auch eine sichere und umweltverträgliche ist. \\
Jedoch stößt diese Problematik in der Bevölkerung bisher leider kaum Resonanz, da das Interesse der allgemeinen Bevölkerung und die Publikation in den Medien fehlt. Diese Aspekte motivierten uns maßgeblich zur Wahl dieses Themas. Hinzuzufügen ist unser Interesse an der Zukunft der Physik und Gesellschaft sowie unsere Neugier über die Auswirkung eines revolutionären Reaktors.\\ 
Der gesellschaftliche Teil stellt die Analyse und Auswertung eines Abbilds der aktuellen gesellschaftlichen Meinung dar. Das zugrundeliegende Abbild wird anhand eine Umfrage erstellt.\todo{besseres Wort} Der wissenschaftliche Teil der Arbeit ist gestützt durch Literatur- und Internetquellen sowie Interviews von fachkundigen Personen im Gebiet der Physik.\\
Der nachfolgende Teil dieser Arbeit beschäftigt sich mit der Geschichte der Energiegewinnung. Hierbei wird  auf verschiedene Vergleichspunkte eingegangen, um eine aussagekräftige Übersicht über die konventionellen Methoden der Energiegewinnung zu ermöglichen. Diese Übersicht ermöglicht Fortschritte innenrhalb der Entwicklung der verschiedenen Formen der Energiegewinnung darzustellen.\\
Einen größeren Teil der Arbeit nimmt die Betrachtung der Fusionreakionb in wissenschaftlicher Hinsicht ein. Zudem wird die Anwendung und Nutzbarmachung dieser Techologie, sprich der Fusionsreaktor beziehungsweise das Fusionskraftwerk, thematisiert.
Ein letzter großer Abschnitt dieser Arbeit beschäftigt sich wiederum mit der aktuellen Energiesituation im Allgemeinen und den Vorteilen der Kernfusion gegenüber anderen Methoden der Energiegewinnung im Genauen und bildet somit einen entscheidenden Teil der Haptsubstanz dieser Arbeit sowohl in theamtischer als auch in sachlicher Hinsicht.\\
Die Kernfusionstechnologie stellt, wie Lesch verdeutlichte, das Tor zu einem neuen, vielleicht noch bedeutungsvollerem, Zeitalter in der Geschichte der Menscheit dar. Somit zielt diese Arbeit auf die Bildung eines Bewusstseins, einer Einstellung gegnüber der Entwicklung und des Fortschritts der Fusionstechnologie ab.
